%%%%%%%%%%%%%%%%%%%%%%%%%%%%%%%%%%%%%%%%%
% Medium Length Professional CV
% LaTeX Template
% Version 3.0 (December 17, 2022)
% This template originates from:
% https://www.LaTeXTemplates.com
%
% Author:
% Vel (vel@latextemplates.com)
%
% Original author:
% Trey Hunner (http://www.treyhunner.com/)
%
% License:
% CC BY-NC-SA 4.0 (https://creativecommons.org/licenses/by-nc-sa/4.0/)
%
%%%%%%%%%%%%%%%%%%%%%%%%%%%%%%%%%%%%%%%%%

%----------------------------------------------------------------------------------------
%	PACKAGES AND OTHER DOCUMENT CONFIGURATIONS
%----------------------------------------------------------------------------------------

\documentclass[
	%a4paper, % Uncomment for A4 paper size (default is US letter)
	11pt, % Default font size, can use 10pt, 11pt or 12pt
]{resume} % Use the resume class

%\usepackage{ebgaramond} % Use the EB Garamond font

%------------------------------------------------

\name{Maxim Lavrenko} % Your name to appear at the top

% You can use the \address command up to 3 times for 3 different addresses or pieces of contact information
% Any new lines (\\) you use in the \address commands will be converted to symbols, so each address will appear as a single line.

\address{\href{https://github.com/maxinimus}{maxinimus (Github)}} % A secondary address (optional)

\address{mlavrenk@purdue.edu \\ eternalbirch@gmail.com} % Contact information

%----------------------------------------------------------------------------------------

\begin{document}

%----------------------------------------------------------------------------------------
%	EDUCATION SECTION
%----------------------------------------------------------------------------------------

\begin{rSection}{Education}
	
	\textbf{Purdue University} \hfill \textit{August 2022 - June 2025} \\ 
	B.S. in Computer Science \hfill Major GPA: 4.0/4.0 \\
	B.S. in Mathematics \hfill Major GPA: 4.0/4.0 \\
	\hspace*{\fill}  CGPA: 3.97/4.0 \\
	Dean's List \& Semester Honors \hfill {\em August 2022 - Present}
	
\end{rSection}

%----------------------------------------------------------------------------------------
%	Coursework
%----------------------------------------------------------------------------------------

\begin{rSection}{Coursework}
The following are some of the notable courses I will have completed by Spring 2024: \\
Introduction to the Analysis of Algorithms
Data Structures
C Programming
Linear Algebra 1 \& 2 
Discrete Mathematics
Object-Oriented Programming
Computer Architecture
Systems Programming

\end{rSection}

%----------------------------------------------------------------------------------------
%	CAREER OBJECTIVE SECTION
%----------------------------------------------------------------------------------------

\begin{rSection}{Career Objective}
  \small{As a university student deeply passionate about artificial intelligence, software engineering, mathematics, and problem-solving, I aim to gain experience in software enginerring and delve into cutting-edge AI applications and research. Drawing from my academic foundation, I aspire to bridge the gap between theoretical concepts and their transformative impact on society.}
\end{rSection}

%----------------------------------------------------------------------------------------
%	PROJECTS SECTION
%----------------------------------------------------------------------------------------

\begin{rSection}{Projects}

	\begin{rSubsection}{}{}{\bf \href{https://github.com/maxinimus/sofia}{Sofia Chatbot} $\mid$ Python, Poe, Whisper }{\hfill \em July 2023}
		\item Allows text/voice input to chat with POE bots. Built using a reverse-engineered POE API client.
		\item Capabilites such as TTS, switching POE bot models, clearing history, importing mp3 and others, as well as looking through history.
		\item Uses a reversed engineered POE API, so support is limited.	
	\end{rSubsection}

%------------------------------------------------

	\begin{rSubsection}{}{}{\bf \href{https://github.com/maxinimus/LaTeX-Matrix-Calculator}{LaTeX Matrix Calculator Website}$\mid$ Flask, numpy, React, Heroku }{ \hfill \em June - July 2023}

		\item Developed a web application to assist LaTeX users working with matrices and linear algebra.
		\item Calculates and provides LaTeX code for finding things such as inverse, REF, and more based on LaTeX code of a matrix.
		\item Built with Flask for the backend and React for the frontend.
		\item Deployed on Heroku (currently unavailable due to hosting costs).
	\end{rSubsection}

%------------------------------------------------

	\begin{rSubsection}{}{}{\bf \href{https://github.com/maxinimus/music-mixer}{Spotify Playlist Mixer} $\mid$ Flask, React, Spotify API}{\hfill \em June - July 2023}
		\item Built a web application that utilizes the Spotify API to help users without spotify premium.
		\item The tool offers shuffling, which enables people to listen to shuffled playlists on their phone without paying for Spotify Premium, as well as reversing a playlist, and more will be added.
		\item Developed using Flask for the backend and React for the frontend.
	\end{rSubsection}

\end{rSection}

%----------------------------------------------------------------------------------------
%	TECHNICAL STRENGTHS SECTION
%----------------------------------------------------------------------------------------

\begin{rSection}{Technical Strengths}

	\begin{tabular}{ @{} >{\bfseries}l @{\hspace{6ex}} l }
		Technical Skills \ & Python, C, C++, Java, React.js, MySQL \\
		Technologies & HTML5, CSS, Latex \\
		Version Control & Github
	\end{tabular}		

\end{rSection}

%----------------------------------------------------------------------------------------
%	EXAMPLE SECTION
%----------------------------------------------------------------------------------------

%\begin{rSection}{Section Name}

	%Section content\ldots

%\end{rSection}

%----------------------------------------------------------------------------------------

\end{document}
